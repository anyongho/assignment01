\chapter{How to use git utility}\label{ch:conclusion}
I using git utility at window OS. So, I using git bash for apply git.

\section{Basic function}
1) Initial configuration\\
First, run the git Bash to register the name and register the email.
\begin{figure} [!h]
	\centering
	\includegraphics[width=0.8\textwidth]{2}
	\caption{initial configuration}
	\label{fig:2}
\end{figure}\\
name register : git config --global user.name"anyongho"\\
email register : git config --global user.email "anyongho11@naver.com"\\
\\
2) Git repository register\\
Second, using git init instruction to register git repository\\
\begin{figure} [!h]
	\centering
	\includegraphics[width=0.8\textwidth]{3}
	\caption{git init}
	\label{fig:3}
\end{figure}\\
\\\\
3) Create test file\\
3rd, Using echo to create test file\\
\begin{figure} [!h]
	\centering
	\includegraphics[width=0.8\textwidth]{4}
	\caption{make test file}
	\label{fig:4}
\end{figure}\\
\\
4) Git add\\
4th, Using git add to the stage section about test file\\
\begin{figure} [!h]
	\centering
	\includegraphics[width=0.8\textwidth]{5}
	\caption{git add}
	\label{fig:5}
\end{figure}\\
\\
5) Git commit\\
5th, Using git commit\\
git commit can check the version of the source code\\
\begin{figure} [!h]
	\centering
	\includegraphics[width=0.6\textwidth]{6}
	\caption{git commit}
	\label{fig:6}
\end{figure}\\
\\
6) Git remote add\\
6th, using git remote add to github URL\\
\begin{figure} [!h]
	\centering
	\includegraphics[width=0.6\textwidth]{7}
	\caption{git remote add}
	\label{fig:7}
\end{figure}\\
\\
7) Git push\\
7th, Using git push to the git-hub\\
By pushing the code, User can change the code as a latest version\\
\begin{figure} [!h]
	\centering
	\includegraphics[width=0.8\textwidth]{8}
	\caption{git push}
	\label{fig:8}
\end{figure}\\

\section{git clone}
Git clone is the way to download from the github source to local repository\\
By using git clone, several users check the latest version of the source code\\
This is how it works\\ 
\begin{figure} [!h]
	\centering
	\includegraphics[width=0.8\textwidth]{9}
	\caption{git clone the code}
	\label{fig:9}
\end{figure}\\\\\\\\\\\\\\\\\\\\\\\\
\section{git branch}
Git Branch is the most valuable function of the git\\
if many developers work together, it is more safer to using different branch rather than master branch\\
After they develop their parts, they can merge the code and remove that branch\\
This is how it works\\ 
\begin{figure} [!h]
	\centering
	\includegraphics[width=0.8\textwidth]{10}
	\caption{git branch function}
	\label{fig:10}
\end{figure}\\
I made the User branch, so that control my code more efficiently